\documentclass{article}

% if you need to pass options to natbib, use, e.g.:
% \PassOptionsToPackage{numbers, compress}{natbib}
% before loading nips_2016
%
% to avoid loading the natbib package, add option nonatbib:
% \usepackage[nonatbib]{nips_2016}

% \usepackage{nips_2016}

% to compile a camera-ready version, add the [final] option, e.g.:
\usepackage[final]{nips_2016}

\usepackage[utf8]{inputenc} % allow utf-8 input
\usepackage[T1]{fontenc}    % use 8-bit T1 fonts
\usepackage{hyperref}       % hyperlinks
\usepackage{url}            % simple URL typesetting
\usepackage{booktabs}       % professional-quality tables
\usepackage{amsfonts}       % blackboard math symbols
\usepackage{nicefrac}       % compact symbols for 1/2, etc.
\usepackage{microtype}      % microtypography

\usepackage{enumitem}

\title{Your Awesome Project Title}

% The \author macro works with any number of authors. There are two
% commands used to separate the names and addresses of multiple
% authors: \And and \AND.
%
% Using \And between authors leaves it to LaTeX to determine where to
% break the lines. Using \AND forces a line break at that point. So,
% if LaTeX puts 3 of 4 authors names on the first line, and the last
% on the second line, try using \AND instead of \And before the third
% author name.

\author{
  Jinyi Lu \\
  Information Networking Institute \\
  \texttt{jinyil@andrew.cmu.edu} \\
  \And
  Name 2 \\
  Department 2\\
  \texttt{id2@andrew.cmu.edu} \\
  \And
  Name 3\\
  Department 3\\
  \texttt{id3@andrew.cmu.edu} \\
}

\begin{document}

\maketitle

% \begin{abstract}
% No abstract.
% \end{abstract}

\textbf{Keywords (optional):} [e.g., computer vision, convnets, segmentation]

\section*{Instructions}

Proposals should be 1-2 pages long (excluding references) and include the following information:
\begin{itemize}[leftmargin=2em]
    \item Project title and list of group members.

    \item Overview of project idea (approximately half a page). Elaborate on the following:
    \begin{itemize}
        \item Why the problem is important.
        \item What challenges you need to solve.
        \item Which datasets you are planning to use.
        \item What metrics you are planning to use to measure your performance.
    \end{itemize}

    \item A short literature survey of 3 or more relevant papers (up to half a page).

    \item A plan of what you wish to accomplish by the second milestone and by the end of the project (preferably, a list of bullet points).
\end{itemize}

% \subsection*{References}

\end{document}
