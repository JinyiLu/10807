
\subsection{Related work}
\label{others}
Traditional methods for object detection usually utilize image features,such as SIFT and HOG. We investigated three methods utilizing deep convolutional network in object detection.
\subsubsection{R-CNN: Rich feature hierarchies for accurate object detection and semantic segmentation}

 The first challenge for object detection is how to implement localizing within an image. The new method proposed in this paper\cite {rcnn} combines CNN with region proposals. So it is called R-CNN. The general detection process is to extract about 2000 region proposals for each input image. Then utilize CNN to compute a fixed length feature for each region proposal. Finally utilize linear SVM to classify each region. 


The second challenge is how to train a large CNN using limited labeled data. This paper proposed to pre-train CNN on a large auxiliary data set, then continually train on a small data set. This method significantly improved the accuracy of objection detection compared with feature learning models. But training is expensive in space and time, and detection is also slow at test time.

\subsubsection{Fast R-CNN}
This paper\cite {fast-rcnn} talked about how to train a detection network faster. R-CNN is very slow because it extracts feature for each region proposal and there are many duplicate computations. The process could be faster if we share computation. 

This paper proposed to modify R-CNN’s architecture by taking an image and multiple regions of interests as input. Region proposal method usually depends on Selective Search. Each region of interest is pooled into a fixed-size feature map and fully connected layers are used to extract features. There two output vectors: softmax probabilities and per-class bounding-box regression offsets.The second one is to reduce mislocalization.




